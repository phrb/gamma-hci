\documentclass[a4paper,11pt]{article} 
\usepackage{times} 
\usepackage[top=10mm, bottom=15mm, left=20mm, right=20mm]{geometry}

%% Escrevendo em português
\usepackage[brazil]{babel}
\usepackage[utf8]{inputenc}

\linespread{1.1} 

\newcommand{\sepitem}{\vspace{0.1in}\item} 
\newcommand{\titulo}{\item \textbf}
\begin {document}
\small{
\title{                                                                                                                                                                                             
{\small                                                                                                                                                                                             
Departamento de Ciência da Computação \hfill IME/USP}\\\vspace{0.1in}                                                                                                                               
MAC0446/MAC5786 - Princípios de Interação Humano Computador
}

\vspace{-0.6in} 
\author{António Martins Miranda}
\vspace{-0.6in}

\date{Atividade 6 - Parte 1: Definição das regras dos jogos}
\maketitle
}
 
\vspace {-0.3in}
\thispagestyle{empty}

\begin{itemize}
\sepitem \textbf{Space Invaders}:

Este jogo será uma adaptação do já conhecido {\it Space Invaders}.
As principais diferenças em relação ao jogo original é que o nosso suportará apenas um jogador e será jogável com um botão.

O jogo consiste em eliminar todos os {\it aliens} antes deles passarem a fronteira que separa os nossos mundos.
Para tal usaremos uma nave com capacidade ilimitada de tiros e mais do que 3 {\it aliens} não podem ultrapassar a fronteira.
A cada {\it alien} que ultrapassar a fronteira a nave perderá uma vida e numa partida teremos no máximo 3 vidas.
Para que o jogador tenha uma noção do grau de perigo que corre essa fronteira deverá ser indicada na tela de forma clara e distinta.

Vamos ter 3 tipos diferentes de {\it aliens} e cada um, quando destruído, nos oferece uma pontuação diferente. Por exemplo, o {\it alien} do tipo B
valerá inicialmente 10 pontos, o {\it alien} do tipo C valerá inicialmente 20 pontos e o {\it alien} do tipo D valerá inicialmente 30 pontos.
Os {\it aliens} estarão organizados em fileiras. Os {\it aliens} das fileiras da frente serão todos do tipo B, nas fileiras centrais teremos os aliens
do tipo C e nas fileiras da dianteira teremos os {\it aliens} do tipo D. O exército dos {\it aliens} será formado por 50\% de {\it aliens} do tipo B, 30\%
de {\it aliens} do tipo C e 20\% de {\it aliens} do tipo C. À cada dois passos em direção à fronteira do exército de {\it aliens} a pontuação dos mesmos sofrerá
um aumento de 5\% dependendo do tipo.

Os movimento da nave do usuário consiste apenas em movimentos laterais, não controlados pelo usuário, ou seja, a nave oscilará continuamente
entre o lado esquerdo e direito da tela. Já exército de {\it aliens} possui movimentos laterais e movimentos frontais. Após 3 movimentos laterais
o exército de aliens executa um movimento para a frente.

As vidas e a pontuação atual do jogador devem ser apresentadas na parte superior da tela.

Em caso de vitória, ou seja, todos {\it aliens} eliminados, deve ser apresentado uma mensagem descritiva da situação ao jogador e o mesmo se aplica em
caso de derrota.

O jogo terá uma variável de pontuação máxima, sempre que o jogador bater essa pontuação máxima deve ser mostrada uma mensagem adequada junto com a
mensagem de derrota ou vitória. O valor da pontuação máxima tem que ser guardado de forma persistente.

O jogador apenas tem controle sobre os tiros da nave.

\sepitem \textbf{Gol}:

Este jogo trata-se de um simulador de marcação de pênaltis. O jogador terá a chance de concretizar 5 pênaltis e para vencer o desafio ele terá que
marcar mais golos que o adversário. O adversário será sempre o computador. O número dos pênaltis concretizados pelo adversário deve ser definido
automaticamente, ou seja, um valor aleatório entre [2 5]. Em caso de empate deve ser oferecido pênaltis extras ao jogador e ao computador até que
ocorra o desempate. A concretização dos pênaltis por parte do computador será determinado de forma aleatória.

O jogo começa com a direção do chute oscilado entre $0^o$ e $180^o$, após a seleção da direção do chute o jogador tem que selecionar a intensidade do mesmo.
Com a direção e intensidade do chute definidos o remate é efetuado automaticamente.

O guarda-redes será uma inteligência artificial que decidirá quando o melhor momento e direção de salto para pegar a bola.

O golo dependerá muito da direção e intensidade do chute(e obviamente pelo guarda-redes).

As informações sobre a marcação dos pênaltis, tanto do jogador como do computador(adversário), devem ser apresentados no topo da tela. A atualização
da marcação dos pênaltis do jogador deve ser instantânea.

Caso o usuário vença, deve ser apresentado uma mensagem de vitória ao jogador e no caso contrário deve ser apresentando uma mensagem de derrota.

\sepitem \textbf{Space Invaders \& Gol}:

Após as mensagens de vitória ou derrota, se o usuário está restringido a apenas um botão, é necessário definir um {\it timer} para a saída do jogo caso ele
não carregue a tempo no botão. Se ele carregar no botão o jogo reiniciado. No caso em que o jogador é capaz de usar dois botões o usuário tem que decidir
se quer continuar a jogar ou sair do jogo. Para tal um botão pode significar ficar no jogo e o outro sair de jogo. Atenção que isso tem que ser apresentado
de forma clara ao jogador.

\end{itemize}
\vfill

\raggedleft
{\sc Maio/2014}

\end{document}
